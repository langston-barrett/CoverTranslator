\documentclass{article}

\usepackage[bookmarks=true,hyperindex=true,colorlinks=true,
  backref=true,pagebackref=false]{hyperref}

\usepackage{color}
\usepackage{alltt}
\usepackage{xspace}
\usepackage{amssymb}
\usepackage[final]{graphicx}

\newsavebox{\fmbox}
\newenvironment{fmpage}[1]
               {\begin{lrbox}{\fmbox}\begin{minipage}{#1}}
               {\end{minipage}\end{lrbox}\fbox{\usebox{\fmbox}}}

\newenvironment{note}
  {\paragraph {Note:}}
  {}

\newenvironment{shellcode}
  {\begin{alltt}\rule{7cm}{1pt}}
  {\rule{7cm}{1pt}\end{alltt}}

\newcommand{\cfop}{{\tt cfop}\xspace}
\newcommand{\haskell}{{\sc Haskell}}

\title{\cfop\ - First Order Theorem Proving for \haskell}
\author{Gr\'egoire Hamon\footnote{\tt hamon@cs.chalmers.se}}

\begin{document}

\maketitle

\section{Installation}

There are two ways to install the tools: from a {\tt tar} file, and
from {\tt CVS}. If you just want to use the tools, the {\tt tar} is
advised as installation should be easier. {\tt cfop} uses external
tools, namely {\tt ghc} and one (or several) first order provers, I'll
briefly cover installation of some these (not {\tt ghc} which you
already have, don't you? a recent version of {\tt ghc} is
required). 

All those information are given for a {\sc Unix} system. The tools
probably work under {\sc Windows} as well, but that has not been
tested, not even installed.

Only basic installation is covered here, if you run into trouble, or
want to change the tools configuration, see section~\ref{sec:usage}.

\subsection{Installing from a {\tt tar} file}
\label{sec:from-tt-tar}
Installation from a {\tt tar} file should be straightforward. You need
to get a file {\tt CoverTranslator-{\it version}.tar.gz}. If you have
access to Chalmers machines, this file can be found in {\tt
/users/cs/group\_proglog/home/Cover/Automatic/}. Then do the following
(example here is using version 0.3, which is the current when writing
this):
\begin{shellcode}
\$ {\color{red}tar -zxf CoverTranslator-0.3.tar.gz}
\$ {\color{red}cd CoverTranslator-0.3}
\$ {\color{red}./configure --prefix={\color{blue}\textit{prefix\_dir}}}
\$ {\color{red}make}
\$ {\color{red}make install}
\end{shellcode} 
where {\tt\it\color{blue}prefix\_dir} is the installation directory
(default: {\tt /usr/local/}). After installation, the programs {\tt
CoverTranslator}, {\tt cfop}, and {\tt hs2agda} are copied into
{\tt{\it\color{blue}prefix\_dir}/bin}.

\subsection{From the {\tt CVS} repository}
Installing from {\tt CVS} shouldn't be much more complicated, but it
needs a recent version of {\tt bnfc} (that is, a {\tt CVS}
version). The {\tt configure} script in the distribution should take
care of checking if {\tt bnfc} can be found, downloading and compiling
it if necessary. However, this requires you have access to the {\tt
bnfc} repository (and enough rights on it). I don't cover the {\sc
Cover} {\tt CVS} repository and its use, that's detailed on the
\href{http://coverproject.org/Wiki/oddmuse.cgi/Cover_CVS_repository}{Wiki}. If
all goes well the following commands should get you up and running:
\begin{shellcode}
\$ {\color{red}cvs co CoverTranslator}
\$ {\color{red}cd CoverTranslator}
\$ {\color{red}autoconf}
\$ {\color{red}./configure --prefix={\color{blue}\textit{prefix\_dir}}}
\$ {\color{red}make}
\$ {\color{red}make install}
\end{shellcode} 
see~\ref{sec:from-tt-tar} for details on {\tt\it\color{blue}prefix\_dir}.

\subsection{Installing First Order Provers}

\subsubsection{Installing {\tt gandalf}}
{\tt gandalf} homepage is: \url{http://sise.ttu.ee/it/gandalf/}. To
install the prover, download the {\tt tar} file ({\tt
gandalf26r1.tar.gz}\footnote{\url{http://sise.ttu.ee/it/gandalf/gandalf26r1.tar.gz}}
when this is written). Installation then goes as follows:
\begin{shellcode}
\$ {\color{red}tar -zxf gandalf26r1.tar.gz}
\$ {\color{red}cd gandalf26}
\$ {\color{red}make}
\end{shellcode} 
This should build the program. Upon completion, you can copy {\tt
GPLGandalf/gandalf} and {\tt GPLGandalf/zchaff} somewhere in your path
(not mandatory, but you'll have to configure {\tt cfop} with the
correct path otherwise).

\subsubsection{Installing {\tt otter}}
{\tt otter} homepage is:
\url{http://www.mcs.anl.gov/AR/otter/}. To install the prover,
download the {\tt tar} file ({\tt
otter-3.3f.tar.gz}\footnote{\url{http://www.mcs.anl.gov/AR/otter/dist33/otter-3.3f.tar.gz}}
when this is written). Installation then goes as follows:
\begin{shellcode}
\$ {\color{red}tar -zxf otter-3.3f.tar.gz}
\$ {\color{red}cd otter-3.3f}
\$ {\color{red}make install}
\end{shellcode} 
This should copy pre-build binaries in {\tt bin/}. You can keep them
here (this will require configuration of {\tt cfop}, or copy {\tt
bin/otter} and {\tt bin/mace2} somewhere in your path.

\section{Short Tutorial: the queue example}

I consider here a very simple example: the queue example\footnote{code
for this example can be found in the distribution in the {\tt
example/queue/} directory.}. The {\sc Haskell} program we are looking
at is the following:
\begin{shellcode}
module Queue where 

type Queue a     = [a]
empty            = []
add x q          = q ++ [x]
isEmpty q        = null q
front  (x:q)     = x
remove (x:q)     = q

type QueueI a    = ([a], [a])
emptyI           = ([],[])
addI x (f,b)     = flipQ (f, x:b)
isEmptyI (f,b)   = null f
frontI  (x:f, b) = x
removeI (x:f, b) = flipQ (f, b)
flipQ ([], b)    = (reverse b, [])
flipQ q          = q

retrieve :: QueueI Integer -> Queue Integer
retrieve  (f,b) = f ++ (reverse b)

invariant :: QueueI Integer -> Bool
invariant (f,b) = null b || not (null f)
\end{shellcode}
It defines two implementations of queues: a simple one (type {\tt
Queue}), and an efficient one (type {\tt QueueI}). The {\tt retrieve}
function takes an efficient queue and returns a corresponding simple
queue.

First, we would like to prove that retrieving the empty efficient
pqueue, we get the empty queue. \cfop reads properties written directly
in {\sc Haskell}, using a special type {\tt Prop}, and operation over
this type defined in the module {\tt Property}\footnote{this module
can be found in the distribution in the {\tt lib/} directory}. Using
those constructions, our property can be written as:
\begin{shellcode}
prop_empty = retrieve emptyI === empty
\end{shellcode}
Properties are {\sc Haskell} values which name is starting with the
prefix {\tt prop\_}, and they should be of type {\tt Prop}. Here, we
use the {\tt (===)} operator, defined by the property module, and
whose type is {\tt (===) :: a -> a -> Prop}. We can now try to prove
the property, assuming the program is saved in a file {\tt queue.hs}:
\begin{shellcode}
\$ {\color{red}cfop queue.hs}
-- empty failed
\end{shellcode}
Not a huge success, it reports that it failed to prove the
property. When given a {\sc Haskell} file, \cfop generates a first
order problem from each property found in the file, tries to prove it
using a first order prover, and reports success or failure. Why can't
it prove such a simple property? {\tt retrieve} is defined by means of
{\tt (++)} and {\tt reverse}, imported from the Haskell prelude, and
for which we don't know anything. We need some lemmas over those to
establish the result. The module {\tt PropPrelude}\footnote{also found
in {\tt lib/} in the distribution} contains properties over some
prelude functions. Looking at it, it contains two properties which we
could use:
\begin{shellcode}
prop_reverse_nil  = (reverse []) ===  []
prop_append_nil_1 = forAll \$ \(\backslash\)xs -> ( xs ++ [] ) ===  xs 
\end{shellcode}
The second one uses another operator from the {\tt Property} module:
{\tt forAll} which takes a function returning a property. To use those
as lemmas in order to prove our property, we import the {\tt
PropPrelude} module, and we hint their use in the definition of the
property using the {\tt using} construction:
\begin{shellcode}
prop_empty = 
    using [ prop_reverse_nil, prop_append_nil_1 ] \$
    retrieve emptyI === empty
\end{shellcode}
Let's try to prove the property once again:
\begin{shellcode}
\$ {\color{red}cfop -l PropPrelude.hs queue.hs}
++ empty proved
\end{shellcode}
That's a more satisfying result. We can get more information on what
the tool is doing by increasing the verbosity level:
\begin{shellcode}
\$ {\color{red}cfop {\color{blue}-v 2} -l PropPrelude.hs queue.hs}
Compiling PropPrelude.hs to core... done
Compiling queue.hs to core... done
Translating to FOL... done
using prover: Gandalf
trying to prove queue_prop_empty.otter
++ empty proved
\end{shellcode}
First, all the files are compiled to {\tt ghc} core language. Then the
file on which we want to prove properties is translated to first order
logic, one file, containing one first order problem is generated for
each property. This file contains the definitions used by the property
and all lemmas specified by the {\tt using} operation. Finally the
prover (here {\tt gandalf} is used to prove the property. We can also
use {\tt otter}:
\begin{shellcode}
\$ {\color{red}cfop -v 1 {\color{blue}-p otter} -l PropPrelude.hs queue.hs}
using prover: Otter
trying to prove queue_prop_empty.otter
++ empty proved
\end{shellcode}
The actual proof found by the prover is saved in a file named {\tt
queue\_prop\_empty.log}. 

Consider now another property: adding an element to an efficient
queue, then retrieving the result is equal to adding this element to
the queue retrieved from the efficient queue. In {\sc Haskell} terms:
\begin{shellcode}
prop_add =
    forAll \$ \(\backslash\)x -> 
    forAll \$ \(\backslash\)b -> 
      retrieve (addI x (xs, b)) === add x (retrieve (xs, b))
\end{shellcode}
Unfortunately, this property can not be proved as is. We need
induction.  We can prove separately the base case and the step case:
\begin{shellcode}
prop_add_base = 
    using [ prop_append_nil_1, prop_append_nil_2, 
            prop_reverse_nil, prop_reverse_cons ] $
    forAll \$ \(\backslash\)x -> 
    forAll \$ \(\backslash\)b -> 
      retrieve (addI x ([], b)) === add x (retrieve ([], b))

prop_add_step = 
    using [ prop_append_assoc, prop_reverse_cons ] $
    forAll \$ \(\backslash\)y  -> 
    forAll \$ \(\backslash\)ys -> 
    forAll \$ \(\backslash\)x  -> 
    forAll \$ \(\backslash\)b  -> 
      retrieve (addI x (y:ys, b)) === add x (retrieve (y:ys, b))
\end{shellcode}
As we can see in this definition, several new lemmas are used here,
these can be found in the {\tt PropPrelude} module. We can now prove
those:
\begin{shellcode}
\$ {\color{red}cfop -l PropPrelude.hs queue.hs}
++ add_base proved
++ add_step proved
++ empty proved
\end{shellcode}

\section{Usage}
\label{sec:usage}

\subsection{Command line options}
The \cfop program recognizes the following options:
\begin{itemize}
\item {\tt -load {\it file}} -- load a library file. Properties
contained in the file can be used as lemmas.
\item {\tt -prover {\it prover}} -- select first order theorem prover to use,
currently {\tt gandalf} and {\tt otter} are supported.
\item {\tt -verbose {\it level}} -- select the verbosity level, that
is an integer, the greater the level, the more verbose the program is
(currently, level 2 is enough to get all informations).
\end{itemize}
Those options can be abbreviated ({\it e.g.} {\tt -l \it file}, {\tt
  -p \it otter}, ...). They override any default options or options
set using a configuration file.

\subsection{Configuration File}
\label{sec:configuration-file}
The program also reads a configuration file in {\tt
\$HOME/.coverrc}. This file should contain assignments of the form
{\tt \it option = value}. An example {\tt coverrc} file is given in
the {\tt lib/} directory of the distribution. The recognized options
are: 
\begin{itemize}
\item {\tt verbose} -- verbosity level,
\item {\tt prover} -- which prover to use ({\tt gandalf} or {\tt otter}),
\item {\tt ghc} -- command to use for {\tt ghc} (if it's not in the path for 
  example),
\item {\tt CoverTranslator} -- command to use for {\tt CoverTranslator},
\item {\tt gandalf} -- command to use for {\tt gandalf},
\item {\tt otter} -- command to use for {\tt otter},
\item {\tt logFile} -- name of the file where logs should be printed.
\end{itemize}

\subsection{Log Files}
\label{sec:log-files}
\cfop records information about what it's doing in a log file, by
default {\tt cfop.log} in the current directory (this can be
overridden in a configuration file, see~\ref{sec:configuration-file}).
If the tool doesn't behave as expected, look into this file, the error
can come from {\tt ghc}, in which case your program has probably a
problem, or can come from {\tt CoverTranslator}, in which case you
have certainly found a bug (that's too be expected in the current
state of the tool).

\section{Frequently Asked Question}
Well, nobody had used it yet, so those are just questions which should
get answered soon (hopefully).

\begin{description}
\item[What about other provers? tptp format?] the tptp backend
doesn't currently work, but should be fixed soon. Adding support for
other provers in the tool should be trivial.


\item[I tried this formula: ... and the tool broke!]  that's expected
in the current state. There are lots of known (or unknown) issues
left, both in the translation to FOL and in the support for
properties. This release is for people to try it and give their
opinion it's not ready for general use yet.

\item[The {\tt PropPrelude} module doesn't contain anything...]
it only contains  what was needed to make the example work. It would be
nice if people could add properties to it (and even better: prove
them, for that of course, you need the definitions from the prelude,
for now you can just put the code and the property in a separate
module, before we find a better way to do that).
\end{description}

\end{document}
